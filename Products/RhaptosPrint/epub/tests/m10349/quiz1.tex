\documentclass[12pt]{article}
\usepackage{times,epsfig,amsmath}
\input{newcommand}
\addtolength{\oddsidemargin}{-.75in}
\addtolength{\evensidemargin}{-.75in}
\addtolength{\textwidth}{1.3in}
\addtolength{\topmargin}{-.9in}
\addtolength{\textheight}{1.5in}
%
\newcounter{quiz}
\setcounter{quiz}{1}
\renewcommand{\thequiz}{\arabic{quiz}}
%
% Change long hypen appearance
%
\def\hlinefill{\leaders\hrule height3pt depth-2.5pt\hfill}
\def\emrule{\thinspace\hbox to .75em{\hlinefill}\thinspace}
%
\makeatletter
%
% Set path for EPSFIG
%
%\define@key{Gin}{figure}{\def\Gfigname{:Figures:#1}}
%\define@key{Gin}{file}{\def\Gfigname{:Figures:#1}}
%
% Problem environment
%
\newcounter{problem}
\renewcommand{\theproblem}{Q\thequiz.\arabic{problem}}
\newcounter{problempart}
\renewcommand{\theproblempart}{\alph{problempart}}
\newcounter{problemsubpart}
\renewcommand{\theproblemsubpart}{\roman{problemsubpart}}
\newenvironment{problems}%
{
\begin{list}%
{\bf\theproblem\hfill}%
{\usecounter{problem}\setlength{\itemindent}{-2em}\setlength{\labelwidth}{0em}}
}%
{\end{list}}
%
\newenvironment{problemparts}%
{\begin{list}%
{\bf(\theproblempart)\hfil}{\usecounter{problempart}}
}%
{\end{list}}
%
\newenvironment{problemsubparts}%
{\begin{list}%
{(\theproblemsubpart)\hfil}{\usecounter{problemsubpart}}
}%
{\end{list}}
\begin{document}
\sloppy
\begin{center}
\large\textbf{Electrical Engineering 241\\
Quiz \Roman{quiz}\\
September 26, 2002}
\end{center}
\par\noindent
One and one-half hour exam.
One 8 1/2"$\times$11" information sheet can be used.
Each problem is given equal weight.
Please sign the pledge when you are finished.
%
\begin{problems}
\item \textbf{A Simple Circuit}\\
You are given the following simple circuit.
\par\noindent\centerline{\epsfig{figure=circuit33.eps}}
\begin{problemparts}
\item
What is the transfer function between the source and the indicated output current?
\item
If the output current is measured to be $\cos(2t)$, what was the source?
\end{problemparts}
%************
\item \textbf{Circuit Detective Work}\\
The left terminal pair of a two terminal-pair circuit is attached to a testing circuit.
The test source $v_{\textrm{in}}(t)$ equals $\sin(t)$.
\par\noindent\centerline{\epsfig{figure=circuit31.eps}}
We make the following measurements.
\begin{itemize}
\item
With nothing attached to the terminals on the right, the voltage $v(t)$ equals $\dfrac{1}{\sqrt{2}}\cos(t+\frac{\pi}{4})$.
\item
When a wire is placed across the terminals on the right, the current $i(t)$ was $-\sin(t)$.
\end{itemize}
\begin{problemparts}
\item
What is the impedance ``seen'' from the terminals on the right?
\item
Find the voltage $v(t)$ if a current source is attached to the terminals on the right so that $i(t) = \sin(t)$.
\end{problemparts}
%***************
\item \textbf{Analog Computers}\\
Because the differential equations arising in circuits resemble those that describe mechanical motion, we can use circuit models to describe mechanical systems.
An ELEC~241 student wants to understand the suspension system on his car.
Without a suspension, the car's body moves in concert with bumps in the road. 
A well-designed suspension system will smooth out bumpy roads, reducing the car's vertical motion.
If the bumps are very gradual (think of a hill as a large but very gradual bump), the car vertical motion should follow that of the road.
The student wants to find a simple circuit that will model how the car's motion.
He is trying to decide between two circuit models.
\par\noindent\centerline{\epsfig{figure=circuit32.eps}}
Here, road and car displacements are represented by the voltages $v_{\textrm{road}}(t)$ and $v_{\textrm{car}}(t)$, respectively.
\begin{problemparts}
\item
Which circuit would you pick? Why?
\item
For the circuit you picked, what will be the amplitude of the car's motion if the road has a displacement given by $v_{\textrm{road}}(t)=1+\sin(2t)$?
\end{problemparts}
%*********
\end{problems}
\end{document}